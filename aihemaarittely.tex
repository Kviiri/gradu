% --- Template for thesis / report with tktltiki2 class ---
% 
% last updated 2013/02/15 for tkltiki2 v1.02

\documentclass[12pt,finnish]{tktltiki2}

% tktltiki2 automatically loads babel, so you can simply
% give the language parameter (e.g. finnish, swedish, english, british) as
% a parameter for the class: \documentclass[finnish]{tktltiki2}.
% The information on title and abstract is generated automatically depending on
% the language, see below if you need to change any of these manually.
% 
% Class options:
% - grading                 -- Print labels for grading information on the front page.
% - disablelastpagecounter  -- Disables the automatic generation of page number information
%                              in the abstract. See also \numberofpagesinformation{} command below.
%
% The class also respects the following options of article class:
%   10pt, 11pt, 12pt, final, draft, oneside, twoside,
%   openright, openany, onecolumn, twocolumn, leqno, fleqn
%
% The default font size is 11pt. The paper size used is A4, other sizes are not supported.
%
% rubber: module pdftex

% --- General packages ---

\usepackage[utf8]{inputenc}
\usepackage[T1]{fontenc}
\usepackage{lmodern}
\usepackage{microtype}
\usepackage{amsfonts,amsmath,amssymb,amsthm,booktabs,color,enumitem,graphicx}
\usepackage[pdftex,hidelinks]{hyperref}

% Automatically set the PDF metadata fields
\makeatletter
\AtBeginDocument{\hypersetup{pdftitle = {\@title}, pdfauthor = {\@author}}}
\makeatother

% --- Language-related settings ---
%
% these should be modified according to your language

% babelbib for non-english bibliography using bibtex
\usepackage[fixlanguage]{babelbib}
\selectbiblanguage{finnish}

% add bibliography to the table of contents
\usepackage[nottoc]{tocbibind}
% tocbibind renames the bibliography, use the following to change it back
\settocbibname{Lähteet}

% spacing
\usepackage{setspace}
\onehalfspacing

% enumitem
\usepackage{enumitem}

% algorithm snippets
\usepackage{caption}

% copy paste from http://tex.stackexchange.com/questions/33866/algorithm-tag-and-page-break
\DeclareCaptionFormat{algor}{%
  \hrulefill\par\offinterlineskip\vskip1pt%
    \textbf{#1#2}#3\offinterlineskip\hrulefill}
\DeclareCaptionStyle{algori}{singlelinecheck=off,format=algor,labelsep=space}

% --- Theorem environment definitions ---

\newtheorem{lau}{Lause}
\newtheorem{lem}[lau]{Lemma}
\newtheorem{kor}[lau]{Korollaari}

\theoremstyle{definition}
\newtheorem{maar}[lau]{Määritelmä}
\newtheorem{ong}{Ongelma}
\newtheorem{alg}[lau]{Algoritmi}
\newtheorem{esim}[lau]{Esimerkki}

\theoremstyle{remark}
\newtheorem*{huom}{Huomautus}


% --- tktltiki2 options ---
%
% The following commands define the information used to generate title and
% abstract pages. The following entries should be always specified:

\title{LCL-ongelmat 4-säännöllisissä toroidihilaverkoissa}
\author{Kalle Viiri}
\date{\today}
\level{Pro Gradu -aihemäärittely}

% The following can be used to specify keywords and classification of the paper:

% classification according to ACM Computing Classification System (http://www.acm.org/about/class/)
% This is probably mostly relevant for computer scientists
% uncomment the following; contents of \classification will be printed under the abstract with a title
% "ACM Computing Classification System (CCS):"

% If the automatic page number counting is not working as desired in your case,
% uncomment the following to manually set the number of pages displayed in the abstract page:
%
% \numberofpagesinformation{16 sivua + 10 sivua liitteissä}
%
% If you are not a computer scientist, you will want to uncomment the following by hand and specify
% your department, faculty and subject by hand:
%
% \faculty{Matemaattis-luonnontieteellinen}
% \department{Tietojenkäsittelytieteen laitos}
% \subject{Tietojenkäsittelytiede}
%
% If you are not from the University of Helsinki, then you will most likely want to set these also:
%
% \university{Helsingin Yliopisto}
% \universitylong{HELSINGIN YLIOPISTO --- HELSINGFORS UNIVERSITET --- UNIVERSITY OF HELSINKI} % displayed on the top of the abstract page
% \city{Helsinki}
%


\begin{document}
% --- Front matter ---

% --- Main matter ---

\mainmatter       % clear page, start arabic page numbering

\section{Hajautettu algoritmi}

Hajautetut algoritmit ovat formaali lähestymistapa erilaisia verkkojen ominaisuuksia koskevien ongelmien ratkaisemiseen. Käyttämässäni LOCAL-mallissa jokainen verkon solmu ajatellaan omana laskentayksikkönään, ja kunkin laskenta-askeleen aikana kukin solmu voi lähettää viestin kullekin naapurisolmulleen ja vastaavasti vastaanottaa viestin kultakin naapuriltaan. Vastaanotettuja viestejä voidaan käyttää osana paikallista laskentaa. Laskentayksiköt voivat asettaa itselleen numeerisen merkin (engl. \textit{label}). Algoritmin suoritus lopetetaan, kun kaikki laskentayksiköt ovat valmiita. Tällöin laskentayksiköiden merkit muodostavat algoritmin tulosteen.

\section{LCL-ongelmat toroidihilaverkossa}

Eräs hajautettujen algoritmien kontekstissa merkittävä verkko-ongelmien osa-alue on paikallisesti tarkistettavat merkinnät (engl. \textit{locally checkable labeling}, jatkossa \textit{LCL}). LCL-ongelmissa yksittäisten solmujen merkin oikeellisuuden voi tarkistaa tutkimalla vain kiinteällä etäisyydellä niistä olevia solmuja. Ongelman voi siis määritellä luettelemalle kullekin eri leimalle sallitut naapurustot tältä säteeltä. Esimerkki tunnetusta LCL-ongelmasta on verkon väritysongelma, jossa kunkin solmun leiman on oltava eri kuin yhdenkään sen naapureista.

Aion käsitellä pro gradu -tutkielmassani erityisesti LCL-ongelmia, joissa tarkasteltava etäisyys on 1, eli kunkin solmun leiman oikeellisuus riippuu vain niiden naapureiden saamista leimoista. Rajaan mahdollisten leimojen joukon binääriseksi, eli kukin solmu voi saada leimakseen vain nollan tai ykkösen. Aion myös rajata tarkasteltavan verkkotyypin 4-säännölliseen toroidihilaan, eli suorakulmiomaiseksi 4-säännölliseksi verkoksi jonka vastakkaiset reunat on kytketty kaarilla toisiinsa.

Erityinen tavoitteeni on lähestyä tutkielmassani synteesiä, eli automatisoitua ratkaisualgoritmien kehittelyä verkko-ongelmiin. Askeleena kohti synteesin tutkimista perehdyn myös yllä kuvattujen LCL-ongelmien luokitteluun vaativuuskategorioihin niiden eri ominaisuuksien perusteella.

\section{Motivaatio}

LCL-ongelmien ratkaisujen paikallinen luonne viittaa siihen, että monet niistä voi myös ratkaista paikallisesti. Hajautetun algoritmin mallissa paikallisista ratkaisuista koostuva kokonaisratkaisu on vakioaikainen. Tutkimalla LCL-ongelmia ja synteesimahdollisuuksia voimme saada uutta tietoa siitä, mitkä ominaisuudet tekevät verkko-ongelmista vaikeita ja mitkä ominaisuudet puolestaan viittaavat ongelmien ratkaistavuuteen vakioajassa.


% --- References ---
%
% bibtex is used to generate the bibliography. The babplain style
% will generate numeric references (e.g. [1]) appropriate for theoretical
% computer science. If you need alphanumeric references (e.g [Tur90]), use
%
% \bibliographystyle{babalpha-lf}
%
% instead.


% --- Appendices ---

% uncomment the following

% \newpage
% \appendix
% 
% \section{Esimerkkiliite}

\end{document}

