% --- Template for thesis / report with tktltiki2 class ---
% 
% last updated 2013/02/15 for tkltiki2 v1.02

\documentclass[12pt,finnish]{tktltiki2}

% tktltiki2 automatically loads babel, so you can simply
% give the language parameter (e.g. finnish, swedish, english, british) as
% a parameter for the class: \documentclass[finnish]{tktltiki2}.
% The information on title and abstract is generated automatically depending on
% the language, see below if you need to change any of these manually.
% 
% Class options:
% - grading                 -- Print labels for grading information on the front page.
% - disablelastpagecounter  -- Disables the automatic generation of page number information
%                              in the abstract. See also \numberofpagesinformation{} command below.
%
% The class also respects the following options of article class:
%   10pt, 11pt, 12pt, final, draft, oneside, twoside,
%   openright, openany, onecolumn, twocolumn, leqno, fleqn
%
% The default font size is 11pt. The paper size used is A4, other sizes are not supported.
%
% rubber: module pdftex

% --- General packages ---

\usepackage[utf8]{inputenc}
\usepackage[T1]{fontenc}
\usepackage{lmodern}
\usepackage{microtype}
\usepackage{amsfonts,amsmath,amssymb,amsthm,booktabs,color,enumitem,graphicx}
\usepackage[pdftex,hidelinks]{hyperref}

% Automatically set the PDF metadata fields
\makeatletter
\AtBeginDocument{\hypersetup{pdftitle = {\@title}, pdfauthor = {\@author}}}
\makeatother

% --- Language-related settings ---
%
% these should be modified according to your language

% babelbib for non-english bibliography using bibtex
\usepackage[fixlanguage]{babelbib}
\selectbiblanguage{finnish}

% add bibliography to the table of contents
\usepackage[nottoc]{tocbibind}
% tocbibind renames the bibliography, use the following to change it back
\settocbibname{Lähteet}

% spacing
\usepackage{setspace}
\onehalfspacing

% enumitem
\usepackage{enumitem}

% algorithm snippets
%\usepackage{algorithm}
% \usepackage{algpseudocode}
%\usepackage{caption}
%\floatname{algorithm}{Algoritmi}

% copy paste from http://tex.stackexchange.com/questions/33866/algorithm-tag-and-page-break
%\DeclareCaptionFormat{algor}{%
%  \hrulefill\par\offinterlineskip\vskip1pt%
%    \textbf{#1#2}#3\offinterlineskip\hrulefill}
%\DeclareCaptionStyle{algori}{singlelinecheck=off,format=algor,labelsep=space}
%\captionsetup[algorithm]{style=algori}

% --- Theorem environment definitions ---

\newtheorem{lau}{Lause}
\newtheorem{lem}[lau]{Lemma}
\newtheorem{kor}[lau]{Korollaari}

\theoremstyle{definition}
\newtheorem{maar}[lau]{Määritelmä}
\newtheorem{ong}{Ongelma}
\newtheorem{alg}[lau]{Algoritmi}
\newtheorem{esim}[lau]{Esimerkki}

\theoremstyle{remark}
\newtheorem*{huom}{Huomautus}


% --- tktltiki2 options ---
%
% The following commands define the information used to generate title and
% abstract pages. The following entries should be always specified:

\title{Verkko-ongelmien paikallisuus toroidihiloissa}
\author{Kalle Viiri}
\date{\today}
\level{Pro gradu -luonnos}
\abstract{Tähän tulee abstrakti.}

% The following can be used to specify keywords and classification of the paper:

\keywords{Hajautettu algoritmi, LOCAL}

% classification according to ACM Computing Classification System (http://www.acm.org/about/class/)
% This is probably mostly relevant for computer scientists
% uncomment the following; contents of \classification will be printed under the abstract with a title
% "ACM Computing Classification System (CCS):"
\classification{
\\\textbf{Computing methodologies---Randomized search}
\\\textit{Computing methodologies--Game tree search}
}

% If the automatic page number counting is not working as desired in your case,
% uncomment the following to manually set the number of pages displayed in the abstract page:
%
% \numberofpagesinformation{16 sivua + 10 sivua liitteissä}
%
% If you are not a computer scientist, you will want to uncomment the following by hand and specify
% your department, faculty and subject by hand:
%
% \faculty{Matemaattis-luonnontieteellinen}
% \department{Tietojenkäsittelytieteen laitos}
% \subject{Tietojenkäsittelytiede}
%
% If you are not from the University of Helsinki, then you will most likely want to set these also:
%
% \university{Helsingin Yliopisto}
% \universitylong{HELSINGIN YLIOPISTO --- HELSINGFORS UNIVERSITET --- UNIVERSITY OF HELSINKI} % displayed on the top of the abstract page
% \city{Helsinki}
%


\begin{document}
% --- Front matter ---

\frontmatter      % roman page numbering for front matter

\maketitle        % title page
\makeabstract     % abstract page

\tableofcontents  % table of contents

% --- Main matter ---

\mainmatter       % clear page, start arabic page numbering

\section{Paikallisuus}

Hajautettu algoritmi on laskennan malli, jossa laskentaprosessi esitetään verkkona, jonka solmut ovat laskentayksiköitä ja kaaret niiden välisiä kytkentöjä. Solmut kommunikoivat keskenään ja yrittävät toisiltaan saamiensa viestien perusteella valita tulosteen joka ratkaisee halutun verkko-ongelman. Esimerkiksi verkon väritysongelmaa ratkaistaessa kukin solmu yrittää valita tulosteen joka on eri kuin kaikilla sen naapureilla.

%todo: päätä haluatko puhua "solmuista" vai "laskentayksiköistä" missäkin kontekstissa ja ole konsistentti

Linial~\cite{linial92} esittelee hajautetun laskennan seuraavasti: Olkoon $G = (V, E)$ verkko jonka solmut ovat laskentayksiköitä ja kaaret laskentayksiköiden välisiä kytkentöjä. Jokaisella aikayksiköllä kukin laskentayksikkö voi lähettää mielivaltaisen paljon dataa naapureilleen, ja suorittaa mielivaltaisen paljon laskentaa omilta naapureiltaan saamallaan datalla.

Koska algoritmin aikavaativuutta arvioidaan pelkästään tiedonvälitykseen kuluvien aikayksikköjen perusteella, on mielekästä tietää millaisiin verkko-ongelmiin on löydettävässä ratkaisu paikallisesti, eli vakioajassa. Toinen mielenkiintoinen kysymys on, milloin ongelman ratkaisuehdotuksen oikeellisuuden voi määritellä paikallisesti, eli tarkistamalla vain vakiosäteisen naapuruston kunkin solmun ympäristöstä~\cite{linial92}.

%TODO: Laajenna kokonainen kappale symmetrian ongelmista yksinkertaistenkin tehtävien kanssa, esittele "hierarkia" anonyymit solmut --> edge coloring/port numbers --> uniikit ID:t

Hajautetun algoritmin laskentavoimaan vaikuttaa myös solmujen saama tieto verkon rakenteesta~\cite{linial92}. Yksinkertaisimmillaan kaikki solmut ovat identtisiä anonyymejä laskentayksikköjä, mikä tekee useista ongelmista mahdottomia ratkaista hajautetulla algoritmilla. Mahdottomuuden syynä on usein solmujen väliset symmetriat, kun kahden täsmälleen identtisessä asemassa olevan solmun tulisi pystyä valitsemaan erilaiset arvot. Tämän vuoksi oletamme jatkossa, että kullakin solmulla on tunnuksenaan uniikki luonnollinen luku.

Verkko-ongelmaa, jossa halutaan määrittää kullekin solmulle jokin arvo sen naapuruston rajoittamana, kutsutaan paikallisesti tarkastettavaksi merkinnäksi (engl. \textit{locally checkable labeling}, jatkossa LCL). Naor ja Stockmeyer~\cite{naor95} määrittelevät LCL-ongelman $\mathcal{L}$ formaalisti nelikkona $(r, \Sigma, \Gamma, \mathcal{C})$ missä $r$ on paikallisuuden määrittävä säde, $\Sigma$ on äärellinen joukko mahdollisia syötearvoja solmuille, $\Gamma$ on äärellinen joukko mahdollisia tulostearvoja, ja $\mathcal{C} \subseteq \Sigma \times \Gamma$ on joukko $r$-säteisiä verkkoja, jotka ovat ongelman sallittuja ratkaisuja.

Verkolle $G = (V, E)$ tehty merkintä $\lambda : V \rightarrow \Sigma \times \Gamma$ on LCL $\mathcal{L}$:n laillinen ratkaisu jos kaikille $u \in V$ vastaa $u$:n $r$-säteinen naapurusto on isomorfinen jonkin $\mathcal{C}$:n verkon kanssa.





% --- References ---
%
% bibtex is used to generate the bibliography. The babplain style
% will generate numeric references (e.g. [1]) appropriate for theoretical
% computer science. If you need alphanumeric references (e.g [Tur90]), use
%
% \bibliographystyle{babalpha-lf}
%
% instead.

\bibliographystyle{babplain-lf}
\bibliography{lahteet}


% --- Appendices ---

% uncomment the following

% \newpage
% \appendix
% 
% \section{Esimerkkiliite}

\end{document}

